\documentclass{article}
\usepackage{footmisc}

\textheight=8\baselineskip
\setlength{\pdfpageheight}{5in}

\newcounter{savefootnote}
\newcounter{symfootnote}
\newcommand{\symfootnote}[1]{%
   \setcounter{savefootnote}{\value{footnote}}%
   \setcounter{footnote}{\value{symfootnote}}%
   \ifnum\value{footnote}>8\setcounter{footnote}{0}\fi%
   \let\oldthefootnote=\thefootnote%
   \renewcommand{\thefootnote}{\fnsymbol{footnote}}%
   \footnote{#1}%
   \let\thefootnote=\oldthefootnote%
   \setcounter{symfootnote}{\value{footnote}}%
   \setcounter{footnote}{\value{savefootnote}}%
}

\makeatletter
  \renewcommand\@makefnmark{%
    \hbox{%
      \ifx\@fnsymbol\FN@fnsymbol@sym
        \expandafter\@firstofone
      \else
        \expandafter\@textsuperscript
      \fi
      {\normalfont\@thefnmark}%
    }%
  }
\makeatother

\begin{document}

I seek concurrent symbolic and arabic-numeral footnotes that follow the \emph{Chicago Manual of Style} 14.24.

A symbolic footnote should be superscript as the in-text reference.\symfootnote{But the symbol at the beginning of the note itself should \emph{not} be superscript and not followed by a period.} Arabic-numeral footnotes should also be superscript as the in-text reference.\footnote{The numeral should not be superscript at the beginning of the note, which should also be followed by a period.}

\end{document}

